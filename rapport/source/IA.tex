\section{Intelligence Artificielle}

\subsection{Stratégies}
    \subsubsection{Intelligence aléatoire}
        Lorsque le personnage est dans une salle de combat, il joue aléatoirement des cartes jusqu'à une éventuelle fin de combat. Il peut aussi choisir de finir son tour. Lorsqu'il est dans la salle de repos, il choisira aléatoirement entre se soigner ou améliorer ses statistiques. Lorsqu'il est dans une salle d'entrainement ou si le joueur à remporté un combat, le personnage choisira aléatoirement une carte parmis les trois proposées pour rajouter à son deck, ou choisira de passer à la salle suivante directement. Si il est à l'exterieur d'une salle, et donc sur la carte, la seule option disponible est de rentrer dans la salle suivante.
    
    \subsubsection{Intelligence basée sur des heuristiques}
        Nous proposons un ensemble d'heuristiques pour chaque salle:
        \begin{itemize}
            \item \textbf{Salle de repos:} Si le joueur a moins de 50\% de sa vie, il choisira de se soigner. Dans le cas contraire il méditera afin de gagner de l'attaque et du block définitifs.
            \item \textbf{Salle d'entrainement:} Dans cette salle, le joueur peut choisir entre 3 cartes. Il y a des cartes considérées comme offensives (les cartes qui attaquent, et/ou débuff les ennemis) et d'autres commes défensives (soin/block ainsi que les buffs). Afin de choisir une carte à rajouter au deck, le joueur va essayer d'avoir le même nombre de cartes offensives que de cartes défensives. Afin que son deck s'améliore, on oblige toujours l'IA a prendre une carte, même quand son deck est plein, et on retire une carte possédant le score le plus faible du deck. Ce score est choisi de manière arbitraire en fonction de toutes les caractéristiques de la carte, de telle sorte que les cartes du deck de départ soient les premières à être supprimées.
            \item \textbf{Salle d'ennemis:} Durant le combat, le joueur effectue un certain nombre d'étapes:
            \begin{itemize}
                \item \textbf{\'Etape 1:} Si le joueur est attaqué et qu'il n'a pas de buff "evade" sur lui, et si le joueur a dans les mains une carde donnant le buff "evade", il va la jouer.\footnote{La carte fournissant le buff "evade" étant unique et la plus coûteuse, elle ne serait pas jouée sinon, bien qu'elle donne un avantage certain.}
                \item \textbf{\'Etape 2:} S'il n'a pas le buff evade appliqué sur lui, il va ensuite essayer de blocker les dégats que vont lui infliger les ennemis, jusqu'à ce qu'il n'ait plus d'énergie suffisante, de carte de block/soin, ou que l'intégralité des dégats des ennemis soient bloqués.
                \item \textbf{\'Etape 3:} S'il en a la capacité, le joueur va ensuite d'attaquer les ennemis. Il va sélectionner le plus faible, et lui faire des dégats s'il peut faire baisser sa vie (et non juste son block). Dans le cas contraire il n'attaquera pas.
                \item \textbf{\'Etape 4:} S'il lui reste de l'énergie après toutes ces étapes, il va jouer le reste de ses cartes aléatoirement.
                \item \textbf{Fin du combat:} En fin de combat, le joueur pourra choisir une carte à ajouter à son deck, il la choisira de la m\^eme manière qu'il choisit des cartes dans la salle d'entrainement.
            \end{itemize}
            L'élément du joueur et des ennemi est ici considéré. L'eau infligera plus de dégat à l'air, lui sur le feu, lui sur la terre et enfin cette dernière sur l'eau. De manière générale lors d'un combat, le joueur joue prioritairement les cartes dont les éléments lui sont défavorables, afin d'essayer de terminer le tour en étant d'un autre élément et ainsi ne pas fausser le calcul du block. Si à l'étape 4, il ne lui reste que des cartes à jouer d'un élément défavorable, il n'en jouera aucune. On n'empêche cependant pas le joueur de jouer des cartes aux étapes 1, 2, 3.
        \end{itemize}
        
        \subsubsection{Intelligence artificielle basée sur des arbres de recherche}
        Cette intelligence artificielle n'interviendra que dans le choix des cartes à jouer. Le reste des commandes effectuées se feront toujours avec le principe de l'ia heuristique (choix des cartes à ajouter au deck, dormir/méditer...).
        \par Jouer des cartes se découpe maintenant en plusieurs étapes:
        \begin{itemize}
            \item Etape 1: conception de l'arbre: l'arbre est construit à partir des cartes contenues dans la main du joueur (au nombre de 5). Toutes les cibles permises sont incluses. Les attributs d'un noeud de l'arbre est: son index, correspondant à l'index de la carte dans la main du joueur (< 0 si il ne correspond pas à une carte), sa cible (sous la forme d'un entityID),un booléen indiquant si la carte a été choisie, et un vector contenant les cartes/noeuds suivants. La racine de l'arbre possède l'index -2, tandis que les noeuds d'index -1 indiquent la commande "passer à l'entité suivante".
            \item Etape 2: parcours de l'arbre et choix de la meilleure combinaison de cartes. Ce choix se fait basé sur un score. On parcourt l'arbre en récursif en clonnant les états. Comme on n'a que 5 cartes à prendre en compte ainsi que la commande passer le tour, ce choix ne semblait pas trop consommateur en mémoire. Le score est calculé à partir de: 
            \begin{itemize}
                \item la vie perdue par le joueur lorsqu'il a joué ses cartes et après calcul des dommages infligés par les ennemis
                \item le nombre d'ennemis tués
                \item le nombre de vie perdue par un ennemi s'il n'est pas tué, pour tous les ennemis
                \item si le joueur est mort après avoir joué toutes ses cartes, imposant le score à $-\infty$
            \end{itemize}
            \item Etape 3: ajout des commandes correspondant aux cartes à jouer dans le moteur, ainsi que la commande fin du tour.
        \end{itemize}
        
    \clearpage
    
\subsection{Conception logiciel}
    Le diagramme des classes pour l’intelligence artificielle est présenté en \autoref{uml:ai}.
    
    \underline{\textbf{AI\_Random}}
    \par Classe implantant l'intelligence aléatoire.
    
    \underline{\textbf{AI\_Heuristic}}
    \par Classe implantant l'IA basée sur une heuristique.
    
    \underline{\textbf{AI\_Deep}}
    \par Classe implantant l'IA basée sur les arbres de recherche.
    
    \underline{\textbf{Node}}
    \par Classe implantant les noeuds de l'arbre de recherche étant utilisé dans la classe IA\_Deep.
    
    \clearpage
    
    \begin{figure}[hp]
    \includegraphics[width=0.6\paperheight]{images/ai.png}
    \caption{\label{uml:ai}Diagramme des classes d'intelligence artificielle.} 
    \end{figure}
    
    \clearpage
    %\begin{landscape}
    %\begin{figure}[p]
    %\includegraphics[width=0.9\paperheight]{ai.pdf}
    %\caption{\label{uml:ai}Diagramme des classes d'intelligence artificielle.} 
    %\end{figure}
    %\end{landscape}
